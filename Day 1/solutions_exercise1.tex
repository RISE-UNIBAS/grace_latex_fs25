\documentclass[a4paper, twocolumn, openany]{book}
\pagenumbering{Roman}

\title{My first book}
\author{Me, myself and I}
\date{\today}

\begin{document}
	\maketitle
	
	\tableofcontents
	
	\chapter{Chapter one}
	
	\noindent Far out in the uncharted backwaters of the unfashionable end of the Western Spiral arm of the Galaxy lies a small unregarded yellow sun.

    Orbiting this at a distance of roughly ninety-eight million miles is an utterly insignificant little blue-green planet whose ape-descended life forms are so amazingly primitive that they still think digital watches are a pretty neat idea.

	\section{The first section}
	
	This planet has—or rather had—a problem, which was this: most of the people living on it were unhappy for pretty much of the time. Many solutions were suggested for this problem, but most of these were largely concerned with the movements of small green pieces of paper, which is odd because on the whole it wasn’t the small green pieces of paper that were unhappy.
	
	And so the problem remained; lots of the people were mean, and most of them were miserable, even the ones with digital watches.
	
	Many were increasingly of the opinion that they’d all made a big mistake in coming down from the trees in the first place. And some said that even the trees had been a bad move, and that no one should ever have left the oceans.
	
	\section{The second section}
	
	And then, one Thursday, nearly two thousand years after one man had been nailed to a tree for saying how great it would be to be nice to people for a change, a girl sitting on her own in a small café in Rickmansworth suddenly realized what it was that had been going wrong all this time, and she finally knew how the world could be made a good and happy place. This time it was right, it would work, and no one would have to get nailed to anything.
	
	Sadly, however, before she could get to a phone to tell anyone about it, a terrible, stupid catastrophe occurred, and the idea was lost for ever.
	
	\paragraph{Oh no!}~\\

\noindent Before we go on to the next chapter,
	
	\begin{quote}
		\small You can try to solve some formatting/typography exercises. Some of them were covered in the slides, but for some you probably need to google around/search on stack overflow to find a solution.\footnote{Note that the size of the quote is smaller than the rest of the text.}
	\end{quote}
	
\noindent So please
	
	\begin{enumerate}
		\item Make a numbered list
		\item Create an unnumbered list within the numbered list, with
		\begin{itemize}
			\item \textbf{the first item bold}
			\item \textsc{and the second in capital letters}
		\end{itemize}
	\end{enumerate}
	
	\chapter{Chapter two}
	
	\noindent This is not her story.
	
	But it is the story of that terrible, stupid catastrophe and some of its consequences.
	
	It is also the story of a book, a book called The Hitchhiker’s Guide to the Galaxy—not an Earth book, never published on Earth, and until the terrible catastrophe occurred, never seen or even heard of by any Earthman.
	
	\section{Another section}
	
	Nevertheless, a wholly remarkable book.
	
	\subsection{With a subsection}
	
	In fact, it was probably the most remarkable book ever to come out of the great publishing corporations of Ursa Minor—of which no Earthman had ever heard either.
	
	\subsection{And another one}
	
	Not only is it a wholly remarkable book, it is also a highly successful one—more popular than the Celestial Home Care Omnibus, better selling than Fifty-three More Things to Do in Zero Gravity, and more controversial than Oolon Colluphid’s trilogy of philosophical blockbusters, Where God Went Wrong, Some More of God’s Greatest Mistakes and Who Is This God Person Anyway?
	
	\section{And yet another section}
	
	In many of the more relaxed civilizations on the Outer Eastern Rim of the Galaxy, the Hitchhiker’s Guide has already supplanted the great Encyclopedia Galactica as the standard repository of all knowledge and wisdom, for though it has many omissions and contains much that is apocryphal, or at least wildly inaccurate, it scores over the older, more pedestrian work in two important respects.
	
	First, it is slightly cheaper; and second, it has the words \textsc{don't panic} inscribed in large friendly letters on its cover.
	
	But the story of this terrible, stupid Thursday, the story of its extraordinary consequences, and the story of how these consequences are inextricably intertwined with this remarkable book begins very simply.
	
	It begins with a house.
	
	\newpage
	
	\chapter*{To be continued \ldots}
		
\end{document}